\chapter{Computing Fundamentals in LabVIEW}
In this chapter, we will go over some of the basic building blocks of a computer program in the context of $\labview$. This chapter will be heavy on examples in $\labview$, but will be light on the trivial details such as how to open a VI and how to place functions and controls. If you have not already done so, review the exercise in section \ref{HowToPlace} in Chapter 1.\\

If you get stuck, you may open the $\labview$ help files my pressing \texttt{Ctrl+?} and navigating to the ``Fundamentals'' section. If you would like to see more examples of how things are done in $\labview$, on the taskbar select ``Help'' and navigate down the menu to find ``Find Examples...'', here you will find a library of examples. These examples range from trivial arithmetic, to programs which would make you a cup of coffee if you supply it with enough hardware.

\section{Decision making structures}
Without the ability to make decisions, computers would not be as useful as they are now. Try to think of a computer program that makes no decisions based on it's input, they do exist and some are even useful. We are not interested in such academic programs, we want our computer to do the heavy lifting for us, just imagine clicking \texttt{yes} or \texttt{no} for a data set of a billion numbers. My computer can do that in $5.76$ seconds (I just checked using $\labview$).\\

If you are familiar with programming, you know that the most fundamental decision making block is an ``if-statement''. You supply the block with a condition, and the program executes statements based on that condition. Technically speaking, $\labview$ has no such thing as an if-statement, it has ``case-structures'' and ``ternary-operators''.
\subsection{Case-Structures}
The case-structure executes a block of code according to its input condition. 


\section{Code looping structures}

\section{Aggregate data types}

\section{Creating functions}

\section{Slightly more advanced Mini Projects}